\section{Introduction}
\label{sec:introduction-shelley}

This document is a formal specification of the functionality of the ledger on the blockchain.
This includes the blockchain layer determining what is a valid block,
but does not include any concurrency issues such as chain selection.
The details of the background and the larger context
for the design decisions formalized in this document are presented
in~\cite{delegation_design}.

In this document,
we present the most important properties that any implementation of the ledger must have.
Specifically, we model the following aspects
of the functionality of the ledger on the blockchain:

\begin{description}
\item[Preservation of value] Every coin in the system must be accounted for,
  and the total amount is unchanged by every transaction and epoch change.
  In particular, every coin is accounted for by exactly one of the following categories:
  \begin{itemize}
    \item Circulation (UTxO)
    \item Deposit pot
    \item Fee pot
    \item Reserves (monetary expansion)
    \item Rewards (account addresses)
    \item Treasury
  \end{itemize}
\item[Witnesses] Authentication of parts of the transaction data by means of
  cryptographic entities (such as signatures and private keys) contained in
  these transactions.
\item[Delegation] Validity of delegation certificates, which delegate
  block-signing rights.
\item[Stake] Staking rights associated to an address.
\item[Rewards] Reward calculation and distribution.
\item[Updates] The update mechanism for Shelley protocol parameters and software.
\end{description}

While the blockchain protocol is a reactive system that is driven by the arrival
of blocks causing updates to the ledger, the formal description is a collection
of rules that compose a
static description of what a \textit{valid ledger} is. A valid ledger state can only
be reached by applying a sequence of inference rules and any valid ledger state
is reachable by applying some sequence of these rules.
The specifics of the semantics we use to define and apply
the rules we present in this document are explained in detail in
\cite{small_step_semantics} (this document will really help the reader's
understanding of the contents of this specification).

The structure of the rules that we give here is such that their application is
deterministic. That is, given a specific initial state and relevant environmental
constants, there is no ambiguity
about which rule should be applied at any given time (i.e.~which state
transition is allowed to take place). This property ensures that the specification
is totally precise --- no
choice is left to the implementor and any two implementations must
behave the same when it comes to deciding whether a block is valid.

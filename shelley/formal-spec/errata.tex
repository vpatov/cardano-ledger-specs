\section{Errata}
\label{sec:errata}

We list issues found within the Shelley ledger which will be corrected in future eras.

\subsection{Total stake calculation}
\label{sec:errata:total-stake}

As described in \cite[3.4.3]{delegation_design}, stake is sometimes considered
relative to all the delegated stake, and sometimes relative to the
totally supply less the amount held by the reserves.
The former is called active stake, the latter total stake.


The $\fun{createRUpd}$ function from Figure~\ref{fig:rules:reward-update} uses the
current value of the reserve pot, named $\var{reserves}$, to calculate the total stake.
It should, however, use the value of the reserve pot as it was during the previous
epoch, since $\fun{createRUpd}$ is creating the rewards earned during the previous epoch
(see the overview in Section~\ref{sec:reward-overview}).

\subsection{Active stake registrations and the Reward Calculation}
\label{sec:errata:active-accounts-and-reward-calc}

The reward calculation takes the set of active reward accounts as a parameter
(the forth parameter of $\fun{reward}$ in Figure~\ref{fig:functions:reward-calc}).
It is only used at the end of the calculation for filtering out unregistered accounts.
Therefore, if a stake credential is deregistered just before the reward calculation starts,
it cannot reregister before the end of the epoch in order to get the rewards.
The time within the epoch when the reward calculation starts should not make any difference.
Note that this does not affect the earnings that stake pools make from their margin,
since each pool's total stake is summed before filtering.
The solution is to not filter by the current active reward accounts, since rewards for
unregistered accounts go to the treasury already anyway.

\subsection{Stability Windows}
\label{sec:errata:stability-windows}

The constant $RandomnessStabilisationWindow$ was intended to be used only
for freezing the candidate nonce in the $\mathsf{UPDN}$ rule
from Figure~\ref{fig:rules:update-nonce}.
This was chosen as a good value for both Ouroboros Praos and Ouroboros Genesis.
The implementation mistakenly used the constant in the $\mathsf{RUPD}$ rule
from Figure~\ref{fig:rules:reward-update}, and used $\StabilityWindow$
in for the $\mathsf{UPDN}$ transition.

The constant $\StabilityWindow$ is a good choice for the $\mathsf{UPDN}$
transition while we remain using Ouroboros Praos, but it will be changed before
adopting Ouroboros Genesis in the consensus layer.

There is no problem with using $RandomnessStabilisationWindow$ for the
$\mathsf{RUPD}$ transition, since anything after $\StabilityWindow$ would work,
but there is no reason to do so.

\subsection{Reward aggregation}
\label{sec:errata:reward-aggregation}

On any given epoch, a reward account can earn rewards by being a member of a stake pool,
and also by being the registered reward account for a stake pool (to receive leader rewards).
A reward account can be registered to receive leader rewards from multiple stake pools.
It was intended that reward accounts receive the sum of all such rewards,
but a mistake caused reward accounts to receive at most one of them.

In Figure~\ref{fig:functions:reward-calc}, the value of $\var{potentialRewards}$ in the
$\fun{rewardOnePool}$ function should be computed using an aggregating union
$\unionoverridePlus$, so that member rewards are not overridden by leader rewards.
Similarly, the value of $\var{rewards}$ in the $\fun{reward}$ function should be computed
using an aggregating union so that leader rewards from multiple sources are aggregated.

This was corrected at the Allegra hard fork.
There were sixty-four stake addresses that were affected,
each of which was reimbursed for the exact amount lost using a MIR certificate.
Four of the stake address were unregistered and had to be re-registered.
The unregistered addresses were reimbursed in transaction
\newline
$a01b9fe136e5703668c9c7776a76cc8541b84824635d237e270670a8ca56a392$,
and the registered addresses were reimbursed in transaction
\newline
$8cab8049e3b8c4802d7d11277b21afc74056223a596c39ef00e002c2d1f507ad$.

\subsection{Byron redeem addresses}
\label{sec:errata:byron-redeem-addresses}

The bootstrap addresses from Figure~\ref{fig:defs:addresses} were not intended
to include the Byron era redeem addresses
(those with addrtype 2, see the Byron CDDL spec).
These addresses were, however, not spendable in the Shelley era.
At the Allegra hard fork they were removed from the UTxO
and the Ada contained in them was returned to the reserves.
